\chapter{Introduction}
\label{sec:introduction}
On-line advertising provides a viable way to support on-line businesses that offer content free of charge to their users, with typical examples being blogs and social networks. Nonetheless, targeted and hence more effective advertising requires that the advertisers have fine-grained information regarding the user's browsing behavior, e.g. pages viewed, searches conducted, products purchased. Techniques that target to record this behavior are designated as \textit{on-line profiling} and have raised significant concerns regarding the protection of the users' sensitive data. Libert~\cite{libert2015exposing} for instance has shown that nearly 9 out of 10 websites leak user data to parties likely unknown to the user.

\emph{Adblockers} aim to improve the user experience and raise the privacy level by eliminating undesired advertising content, as well as preventing the leakage of sensitive user information towards third-party servers. Because of their critical character in terms of privacy, much attention has been paid to the evaluation and comparison of the efficiency of different adblockers~\cite{pujol, ruffel2015, mayer, englehardt} available for the major Internet browsers. The existing literature has often employed the number of blacklisted domains, cookies or HTTP requests as a privacy index~\cite{butkiewicz, pujol, kontaxis}. However, we observe that these metrics do not take into account neither legal entities (i.e., the companies behind third party domains) nor the temporal aspect of the adblocker filtering performance. Furthermore, although the widespread usage of mobile devices has markedly changed the landscape of web browsing over the last years gaining more and more ground against desktop devices, little focus has been laid on the web privacy for mobile devices.

In this thesis we address this problem by providing a quantitative methodology to compare adblocker filtering performance with respect to the filtering of third parties. To this end we define a set of privacy metrics capable of capturing the adblocker's filtering performance and the implications that it brings to the user privacy. We evaluate 12 different browser profile configurations, capturing different adblocker instances, as well as mobile and desktop user clients. Furthermore, we study the adblocker filtering performance over a time-frame of three weeks in order to assess the temporal stability and robustness of their blocking capabilities. Finally, we capture the legal corporations as well as the geographical locations behind the third parties.

Our results show that the usage of adblockers achieves a significant improvement in terms of user privacy. In more detail, the Ghostery adblocker consistently performs better than AdblockPlus when both adblockers are set to the maximum-protection level. In the default settings however, Ghostery does not block third-party. Our results show that the usage of the \textit{do not track} header prevents the loading of a few third parties. We do not observe a significant difference between mobile and desktop clients with respect to the number of loaded third parties.

Our contributions in this thesis can be summarized as follows:
 \begin{itemize}
 \item We provide a quantitative methodology to objectively compare the filtering performance of web adblockers.
 \item We capture the temporal evolution of adblocker filtering performances and study the differences between mobile and desktop devices, as well as the impact of the \emph{do not track} header. Our methodology further allows for the measurement of the influence of other parameters (e.g. third-party cookies) on adblocker filtering performance.
 \item Beyond the domain of the third parties, our model takes into account the underlying legal entities as well as their corresponding geographical locations.
 \item Using our model, we quantify the privacy of 12 different adblocker browser profile configurations over 1000 different Web sites for repetitive daily measurements over the duration of three weeks and discuss the implications in terms of user protection.
\end{itemize}

The remainder of the thesis is organized as follows. In Chapter~\ref{sec:background} we illustrate the objective and functionality of adblockers, while in Chapter~\ref{sec:privacy_metrics} we outline our privacy metrics. Chapter~\ref{sec:evaluation} discusses the experimental setup and the results. Chapter~\ref{sec:related_work} presents the related work and Chapter~\ref{sec:conclusions} summarizes our work.
