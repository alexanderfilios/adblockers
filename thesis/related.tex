\chapter{Related work}
\label{sec:related_work}

\textbf{Tracking techniques:} A much work has already been done towards the investigation of tracking techniques, such as canvas fingerprinting~\cite{mowery2012pixel, acar2014web}, evercookies~\cite{acar2014web}, cookie respawning~\cite{acar2014web}, JavaScript performance and implementations~\cite{mulazzani2013fast, mowery2011fingerprinting, nikiforakis2012you} or Flash-plugin information~\cite{eckersley2010unique, soltani, ayensonflash}.

\textbf{Privacy concerns:} Also, a of research so far has been dedicated to the privacy concerns as a consequence of tracking and fingerprinting by third-party domains~\cite{barford, englehardt, krishnamurthy_privacy_diffusion, nikiforakis, soltani, libert2015exposing}. Castelluccia \textit{et al.}~\cite{castelluccia} showed that the user's interests can be inferred by the ads they receive and their whole profile can be reconstructed. This can lead to discriminations of the users according to their profile details and configurations, as shown in~\cite{mikians, datta}.

\textbf{Countermeasures:} As a result, several methods have been proposed that enable targeted advertisements without compromising user privacy~\cite{adnostic, privad, nurikabe, haddadi, juels, androulaki}. Additionally, there have been a lot of attempts for the detection of tracking behavior and ad-blocking blacklist enhancements~\cite{ma, gugelmann, tran}, while some studies have proposed further mitigation techniques~\cite{roesner, kontaxis}.

\textbf{Comparison of mitigation-techniques:} Since our work focuses in the comparison of the ad-blocking tools, it is useful to present in more detail the work done so far in this field.

Balebako \textit{et al.}~\cite{balebako} propose a method to measure behavioral targeting and the effect of privacy-protection techniques ---e.g. disabling of third-party cookies, Do-Not-Track header, ad-blocking tools--- in the limitation of the behavioral-targeted character of the advertising content, while Krishnamurthy \textit{et al.}~\cite{krishnamurthy_measuring_privacy_loss} compare different privacy-protection techniques against the trade-offs between privacy and page quality. From a different perspective, Leon \textit{et al.}~\cite{leon} investigate and compare the usability of some existing tools designed to limit advertising.

Pujol \textit{et al.}~\cite{pujol} aim to infer the use or no use of an adblocker by examining the HTTP(S) requests sent by a browser, using the ratio of the ad requests and the downloads of filter lists as indicators. Moreover, the filtering performance of 7 different browser profiles ---adblocker-configuration combinations--- is compared based upon the total number of unblocked requests per browser profile. Furthermore, although the ad traffic is examined through the analysis of the number of requests at different time instances throughout the day, the content-type of the ad requests, as well as the effect of enabling non-intrusive ads, no long-term data is collected regarding the tracking behavior of the third parties.

Ruffell \textit{et al.}~\cite{ruffel2015} analyze the effectiveness of various browser add-ons in mitigating and protecting users from third-party tracking networks. In total 7 browser profiles are created, each with a different combination of multiple add-ons and browser settings, and each of the profiles visits the 500 top Alexa Rank websites, while the HTTP request data is recorded with the use of Mozilla Lightbeam. The data is collected for one crawling cycle and followingly the efficiency analysis is performed based upon various graph metrics. Nevertheless, none of these browser profiles examines the difference of the effect of the third-party tracking on devices with Mobile User Agents.

Mayer and Mitchell~\cite{mayer} implemented the tool FourthParty ---an open-source platform for measuring dynamic web content--- as an extension to Mozilla Firefox. Afterwards, they created several browser profiles, so as to test the efficiency of different ad-blocking tools under certain settings (blacklists) and crawled the 500 top Alexa websites three times using FourthParty, in order to extract the average decrease in tracking with the use of the ad-blocking tools. The results of such an analysis may, however, be subject to biases, since the URL sample set used has been vastly examined in the bibliography so far and many adblockers may have been optimized to provide a higher efficiency when tested against it.

Englehardt and Narayanan~\cite{englehardt} use OpenWPM~\cite{englehardt_open_wpm}, a web privacy measurement platform that can simulate users, collect data and record observations, e.g. response metadata, cookies and behavior of scripts. They introduce the ``prominence'' metric to rank third parties
%according to the frequency with which a user will encounter a given third party
%run their evaluation on the 1M Alexa Top Rank
and describe its relationship with their rank, i.e.\ the absolute number of first parties they appear on (degree). They further test the effectiveness of Ghostery, as well as of the third-party-cookie-blocking option in terms of privacy and show that Ghostery's effectiveness drops for the less prominent third-party trackers. Moreover, they use stateful measurements (the browser's profile is not cleared between page visits) and investigate how many third parties are involved in cookie syncing.
%quantify the impact of trackers and third parties on HTTPS deployment (test the hypothesis that third parties impede HTTPS adoption)
Finally, they investigate different fingerprinting techniques, build a detection criterion for each of them and perform measurements to show that the user's behavior is more likely to be fingerprinted on more popular sites.

Their analysis though is not concentrated on the ad-blocking-software effectiveness and has not taken into consideration different combinations of adblockers and settings. Additionally, the filtering-performance evaluation consists in the mere presentation of the most prominent third-party trackers when Ghostery is enabled, while no relevant graph analysis has been performed.