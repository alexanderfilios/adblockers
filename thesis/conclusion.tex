\chapter{Conclusions}
\label{sec:conclusions}
The emerging trend of web advertising as well as the earning potential that it has to offer have turned it into the driving force for the development of a broad spectrum of websites and businesses. Nevertheless, fingerprinting and online-profiling practices that are deployed with main objective to optimize the efficiency of the web advertisements put the protection of the personal information of the end-user at stake and are, hence, in direct conflict with their privacy matters.

Adblockers, on the other hand, aim to counter these risks by removing advertising content and preventing third-party tracking.

Our analysis in the present thesis provides a quantitative methodology to compare the filtering performance of different adblockers. After the examination of multiple browser profiles --- i.e.\ combinations of ad-blocking software and configurations --- for desktop and mobile devices, our results indicate that the usage of an adblocker can indeed increase the privacy level and restrain the leakage of information concerning the browsing behavior of the user towards third-party trackers. Furthermore, we show that the most important factor that can determine the achieved privacy level is according to our experiments the selection of blacklists, whilst the activation of the \textit{do not track} HTTP header only has a minor effect. Our findings suggest that the best-performing adblockers are Ghostery and then AdblockPlus, when both are set to a maximal-protection level, whilst the highest privacy risks exist when no adblocker or Ghostery with its default blacklist settings is used. Finally, our methodology allows for a quantitative evaluation and comparison of any new web ad-blocking software.