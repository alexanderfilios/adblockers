\begin{abstract}
Web advertisements have become an integral part of today's web browsing experience, since they financially support countless websites. Nevertheless, effective advertising requires behavioral targeting, user tracking and profile fingerprinting that raise serious privacy concerns. To counter privacy issues and enhance usability, adblockers emerged as a popular method to filter web requests that do not serve the website's main content. Notwithstanding their popularity, little work has focused so far on the quantification of the privacy level that an adblocker achieves.

In this thesis, our main objective is to propose a quantitative methodology for the objective comparison of adblockers. We base our methodology and model on a set of privacy metrics that capture not only the technical web architecture, but also the underlying corporate component. Our model allows us to examine any combination of ad-blocking software and browser settings and applies both to desktop and mobile clients. More importantly, we capture the temporal aspect of the web infrastructure dynamics.

Our results highlight a significant difference among adblockers in terms of filtering performance, highly depending on the configurations applied. Besides the ability to judge over the filtering capabilities of existing adblockers, our work provides an organized framework to evaluate new adblocker proposals in the future.
\end{abstract}